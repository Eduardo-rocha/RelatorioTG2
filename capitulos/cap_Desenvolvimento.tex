%TCIDATA{LaTeXparent=0,0,relatorio.tex}
                      

\chapter{Desenvolvimento}\label{CapDesenvolvimento}

% Resumo opcional. Comentar se n�o usar.
\resumodocapitulo{Resumo opcional.}

\section{Introdu\c{c}\~{a}o}

Na introdu\c{c}\~{a}o dever\'{a} ser feita uma descri\c{c}\~{a}o geral da
metodologia que foi seguida para o desenvolvimento. A seguir, \'{e} feita a
desci\c{c}\~{a}o do sistema desenvolvido. 

Deve-se ressaltar que equa\c{c}\~{o}es fazem parte do texto, devendo receber pontua\c{c}\~{a}o apropriada e ser numerada. Alguns exemplos s\~{a}o mostrados na se\c{c}\~{a}o \ref
{SecClassificador}.

\section{Arquitetura geral}

\section{Classificador estat\'{i}stico de padr\~{o}es}\label{SecClassificador}

O classificador autom\'{a}tico de padr\~{o}es utiliza o princ\'{i}pio da
menor dist\^{a}ncia no processo de associa\c{c}\~{a}o de dados. Assim, sendo 
$\mathbf{x}$ o vetor de caracter\'{i}sticas extra\'{i}das de uma imagem e $%
\mathbf{P}_{\mathbf{x}}$ sua matriz de covari\^{a}ncias respectiva,
utiliza-se 
\begin{equation}
d_{i}=(\mathbf{x}-\mathbf{p}_{i})^{T}(\mathbf{P}_{\mathbf{x}}+\mathbf{P}_{%
\mathbf{p}_{i}})^{-1}(\mathbf{x}-\mathbf{p}_{i})  \label{EqDistMahalanobis}
\end{equation}
como m\'{e}trica para a dist\^{a}ncia estat\'{i}stica de  $\mathbf{x}$ e um
padr\~{a}o de caracter\'{i}sticas $\mathbf{p}_{i}$ e matriz de covari\^{a}%
ncias $\mathbf{P}_{\mathbf{p}_{i}}$. Esta m\'{e}trica \'{e} conhecida tamb%
\'{e}m pela denomina\c{c}\~{a}o ``dist\^{a}ncia de Mahalanobis''. A dist\^{a}%
ncia definida pela Eq. (\ref{EqDistMahalanobis}) segue distribui\c{c}\~{a}o $%
\chi _{n}^{2}$, em que $n$ \'{e} a dimens\~{a}o da base do vetor $\mathbf{x}$%
. Assim sendo, no caso espec\'{i}fico de $n=3$, o padr\~{a}o associado ao
vetor $\mathbf{x}$ \'{e} dito casado com o padr\~{a}o $\mathbf{p}_{i}$ com $%
5\%$ de margem de erro se 
\begin{equation}
d_{i}\leq 7,815.
\end{equation}

A seguir � mostrado como o modelo Latex apresenta os comandos $\backslash$section $\backslash$subsection e $\backslash$subsubsection. Por quest�o de estilo, o texto deve ser organizado de modo a se evitar o uso de $\backslash$subsubsection. 

\section{Se��o}

Meu texto da se��o.

\subsection{Sub-se��o}

Meu texto da sub-se��o.

\subsubsection{Sub-sub-se��o}

Meu texto da sub-sub-se��o.

Se necess�rio, use notas de rodap� \footnote{Essa � uma nota de rodap�.}

