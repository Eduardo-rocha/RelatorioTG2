%TCIDATA{LaTeXparent=0,0,relatorio.tex}

\resumo{Resumo}{O trabalho consiste do projeto e fabrica��o de uma bomba de inje��o automatizada para a inje��o de contraste radiopaco em neuroangiografia 4D. A m�quina foi constru�da para uso no departamento de f�sica m�dica no Wisconsin Institute for Medical Research (WIMR). 
O controle do injetor ocorre atrav�s de um Controlador L�gico Program�vel (CLP) com tela LCD touch, na qual a interface de usu�rio programada � exibida. Encoders de alta resolu��o foram acoplados a dois atuadores lineares el�tricos DC presentes no injetor, que junto a sensores �pticos d�o retorno da posi��o dos atuadores ao controlador. Acoplou-se tamb�m sensores de press�o para garantir que a press�o de inje��o nunca ultrapasse uma press�o m�xima definida pelo usu�rio. Implementou-se diversos perfis de inje��o onde o fluxo de fluidos pode seguir sinais senoidais, ondas quadradas, sinais de simula��o de fluxo sangu�neo, sinais de ru�do branco e sinais de frequ�ncia variada. In�meros testes foram realizados no WIMR para garantir o funcionamento correto do injetor. Uma documenta��o detalhada foi escrita para dar suporte ao usu�rio e facilitar poss�veis altera��es na m�quina.}

\vspace*{2cm}

\resumo{Abstract}{The project consists of the design and manufacture of an automated injection pump for injection of contrast agents in 4D neuroangiography. The machine was built for use in the department of medical physics at the Wisconsin Institute for Medical Research.
The control of the injector occurs through a Programmable Logic Controller with a LCD touch screen, in which the designed user interface is displayed. High resolution encoders were coupled to two linear DC electric actuators present in the injector, which together with optical sensors give position feedback from the actuators to the controller. Pressure sensors were also coupled to ensure that the injection pressure will never exceed a user-defined maximum pressure. Several injection profiles were implemented where the fluid flow can follow sinusoidal signals, square wave signals, blood flow simulation signals, white noise signals and variable frequency signals. Numerous tests have been performed on the WIMR to ensure correct operation of the injector. Detailed documentation was written to support the user and facilitate possible changes to the machine.}