%TCIDATA{LaTeXparent=0,0,relatorio.tex}

\resumo{Resumo}{O trabalho consiste do projeto e fabrica��o de uma bomba de inje��o automatizada para a inje��o de contraste radiopaco em neuroangiografia 4D. A m�quina foi constru�da para uso no departamento de f�sica m�dica no Wisconsin Institute for Medical Research (WIMR). 
O controle do injetor ocorre atrav�s de um Controlador L�gico Program�vel (CLP) com tela LCD touch, na qual a interface de usu�rio programada � exibida. Encoders de alta resolu��o foram acoplados a dois atuadores lineares el�tricos DC presentes no injetor, que junto a sensores �pticos d�o retorno da posi��o dos atuadores ao controlador. Acoplou-se tamb�m sensores de press�o para garantir que a press�o de inje��o nunca ultrapasse uma press�o m�xima definida pelo usu�rio. Implementou-se diversos perfis de inje��o onde o fluxo de fluidos pode seguir sinais senoidais, ondas quadradas, sinais de simula��o de fluxo sangu�neo, sinais de ru�do branco e sinais de frequ�ncia variada. In�meros testes foram realizados no WIMR para garantir o funcionamento correto do injetor. Uma documenta��o detalhada foi escrita para dar suporte ao usu�rio e facilitar poss�veis altera��es na m�quina.}

\vspace*{2cm}

\resumo{Abstract}{The same as above, in english.}